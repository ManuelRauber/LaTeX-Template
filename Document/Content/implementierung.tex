\chapter{Implementierung}

\section{ToDo}

Um ToDos anzuzeigen können alle Befehle des Packages \enquote{todonotes}\footnote{\url{http://www.ctan.org/tex-archive/macros/latex/contrib/todonotes/}} genutzt werden.
Zur Einfachheit können folgende Befehle genutzt werden:

\begin{verbatim}
\mytodo{Kapitel über Krokodile schreiben!}
\mytodoimprove{Neue Quellen raussuchen!}
\end{verbatim}

\mytodo{Kapitel über Krokodile schreiben!}
\mytodoimprove{Neue Quellen raussuchen!}

\section{Literaturverzeichnis}

Das Literaturverzeichnis generiert die Einträge nach DIN 1502-2. Die Styles sind in der alphadin.bst\footnote{\url{http://mirror.unl.edu/ctan/biblio/bibtex/contrib/german/din1505/alphadin.bst}} zu finden.

Beispiele \cite{web:wiki:latex,book:komascript}

\section{Verzeichnisse}

Dieses Template erstellt einige Verzeichnisse, wie sie entsprechend der DHBW-Richtlinien benötigt werden: Inhaltsverzeichnis, Abbildungsverzeichnis, Tabellenverzeichnis, Abkürzungsverzeichnis und Codeverzeichnis

Zur Demonstration einige Beispiele:

\begin{figure}[H]
\centering
\includegraphics[width=\textwidth]{Images/placeholder.png}
\caption{Ich bin ein tolles Bild!}
\end{figure}

\begin{table}[H]
\centering
\begin{tabular}{ll}
	\textbf{Kopfspalte 1} & \textbf{Kopfspalte 2} \\ \hline\hline
	Ich bin               & eine tolle Tabelle!   \\ \hline
	Bin ich               & nicht wundervoll?
\end{tabular}
\caption{Eine tolle Tabelle}
\end{table}

\section{Source code}

Gerade im IT-Bereich kommt es oft vor, dass Source eingebunden werden soll. Natürlich schön mit Syntax-Highlighting. 
Dieses Template unterstütz dies mit Hilfe von minted\footnote{\url{http://www.ctan.org/tex-archive/macros/latex/contrib/minted/}} und pygments\footnote{\url{http://pygments.org/}}. Die Einrichtung dafür ist beschrieben unter \url{http://www.manuel-rauber.com/post/2013/10/28/My-LaTeX-Environment}.
Des Weiteren ist das Template in der Lage, lange Zeilen umzubrechen\footnote{Allerdings hat die Funktion noch einen Bug: \url{http://tex.stackexchange.com/questions/129383/break-lines-in-minted-code}. Bei bereits eingerücktem Code werden die \enquote{Umbruchpfeile} falsch dargestellt.}, sowie Captions und Boxen anzuzeigen. Zu lange Code-Stücke werden automatisch auf die nächste Seite umgebrochen, was man auch visuell sehen kann. Hier für ein längeres Code-Beispiel:

\begin{listingsbox}{mycsharp}{Beispiel-Code}
using System;
using System.IO;
using System.Net;
using System.Net.Http;
using System.Net.Http.Formatting;
using System.Net.Http.Headers;
using System.Text;
using System.Threading.Tasks;

namespace ShortUrl.Core.Formatters
{
	public class TextMediaTypeFormatter : MediaTypeFormatter
	{
		public TextMediaTypeFormatter()
		{
			SupportedEncodings.Add(Encoding.UTF8);
		}

		public override Task WriteToStreamAsync(Type type, object value, Stream writeStream, HttpContent content,
			TransportContext transportContext)
		{
			return Task.Factory.StartNew(() =>
			{

				var str = (string)value;

				using (var textWriter = new StreamWriter(writeStream))
				{
					textWriter.Write(str);
				}
			});
		}

		public override bool CanReadType(Type type)
		{
			return false;
		}

		public override bool CanWriteType(Type type)
		{
			return typeof(string) == type;
		}
	}
}
\end{listingsbox}

\section{Glossar}

Der Aufbau eines Glossar wird entsprechend durch das Package \enquote{glossaries}\footnote{\url{http://www.ctan.org/tex-archive/macros/latex/contrib/glossaries/}} unterstützt.

Du wolltest schon immer mal wissen, was ein \gls{Instrument} ist? ;-)
