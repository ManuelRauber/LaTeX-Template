% Präambel
% Angepasste Vorlage der DHBW Karlsruhe (http://zil.dhbw-karlsruhe.de/wiki/index.php/LaTeX/Vorlagen)
% Achtung: Ziemlich chaotisch, wird irgendwann mal aufgeräumt :-)

\documentclass[12pt,a4paper,oneside,
final, 
titlepage, 						% Titlepage-Umgebung statt \maketitle
headsepline, 					% horizontale Linie unter Kolumnentitel
%abstracton,					% Überschrift beim Abstract einschalten, Abstract muss dazu in{abstract}-Umgebung stehen 
%DIV11,							% auskommentieren, um den Seitenspiegel zu vergrößern
BCOR6mm,						% Bindekorrektur, die den Seitenspiegel um 6mm nach rechts verschiebt,
toc=listof,
parskip=full,
ngerman,
]{scrreprt}		
\usepackage{morewrites}			% Mehr Write-Streams für LaTeX
\usepackage{tabularx}			% Bessere Tabellenunterstützung
\usepackage{ucs} 				% Dokument in utf8-Codierung schreiben und speichern
\usepackage[utf8x]{inputenc} 	% ermöglicht die direkte Eingabe von Umlauten
\usepackage[ngerman]{babel} 	% deutsche Trennungsregeln und Übersetzung der festcodierten Überschriften
\usepackage[T1]{fontenc} 		% Ausgabe aller zeichen in einer T1-Codierung (wichtig für die Ausgabe von Umlauten!)
\usepackage{graphicx}  			% Einbinden von Grafiken erlauben
\usepackage{amsmath}
\usepackage{mathpazo} 			% Einstellung der verwendeten Schriftarten
\usepackage{textcomp} 			% zum Einsatz von Eurozeichen u. a. Symbolen
\usepackage{xcolor} 			% einfache Verwendung von Farben in nahezu allen Farbmodellen
\usepackage[babel,german=quotes]{csquotes}	% \enquote
\usepackage{setspace}			% Spacing
\usepackage[all]{nowidow}		% Keine Hurenkinder und Schusterjungen
\usepackage{float}
\usepackage[format=plain,indention=0.5cm,hypcap=true]{caption} % Bessere Captions
\usepackage{paralist}			% Bessere List Environment
\usepackage{varioref}			% Ref
\usepackage{hyperref}			
\usepackage{translator}
\usepackage[toc,acronym]{glossaries} 	% zur Erstellung des Abkürzungsberzeichnisses
\usepackage[nameinlink]{cleveref}
\usepackage[nottoc]{tocbibind}
\usepackage[german,colorinlistoftodos]{todonotes}
\usepackage{pdfpages}
\usepackage{array}
\usepackage{booktabs}
\usepackage{subfigure}
\usepackage{enumitem}
\usepackage{algorithm}
\usepackage{algpseudocode}
\usepackage{tikz}
\usetikzlibrary{arrows, shapes, decorations, shapes.geometric, positioning, calc}
\usepackage{ifthen}
\usepackage{xstring}
%\usepackage{tikz-uml} % Doku: http://perso.ensta-paristech.fr/~kielbasi/tikzuml/doc/tikzuml-v0.9.9.pdf
\usepackage[space]{grffile}
\usepackage[automark]{scrlayer-scrpage}
\usepackage{multirow}
\usepackage{ifdraft}
\usepackage[many,minted]{tcolorbox}
\ifdraft{\usepackage{refcheck}}
\usepackage{placeins}

% spacing von matheumgebungen

\setlength{\jot}{10pt}

% Persönliche Daten:

\newcommand{\titel}{Titel der Arbeit}
\newcommand{\untertitel}{Untertitel der Arbeit}
\newcommand{\arbeit}{Typ der Arbeit (z.B. Praxisbericht)}
\newcommand{\studiengang}{Studiengang}
\newcommand{\autor}{Max Mustermann}
\newcommand{\matrikelnr}{007007}
\newcommand{\kurs}{Kursnummer}
\newcommand{\bearbeitungszeitraum}{01.01.2013 - 31.03.2013}
\newcommand{\firma}{Musterfirma GmbH, Karlsruhe}
\newcommand{\abgabe}{31. März 2013}
\newcommand{\betreuerdhbw}{Prof. Erika Mustermann}
\newcommand{\betreuer}{Markus Mustermann}

\newcommand{\jahr}{2013}

% Abkürzungen
\newcommand{\ua}{\mbox{u.\,a.\ }}
\newcommand{\zB}{\mbox{z.\,B.\ }}
\newcommand{\bs}{$\backslash$}
                       
% Eigenes Ref
\newcommand{\myref}[1]{vgl. \cref{#1} \nameref{#1} \vpageref{#1}}
\newcommand{\mylistingref}[1]{vgl. \cref{#1} \vpageref{#1}}

% Todogeschichten
\newcommand{\mytodo}[1]{\todo[color=red,inline]{#1}}
\newcommand{\mytodoimprove}[1]{\todo[color=green,inline]{#1}}

% inline code
\newcommand{\myinlinecode}[1]{\footnotesize\texttt{#1}\normalsize}

% Minted Einstellungen %

\newcommand{\mynewminted}[3]{%
  \newminted[#1]{#2}{#3}%
  \tcbset{myminted/#1/.style={minted language=#2,minted options={#3}}}}

\mynewminted{mycsharp}{csharp}{tabsize=2,fontsize=\footnotesize}
\mynewminted{myjavascript}{js}{tabsize=2,fontsize=\footnotesize}
\mynewminted{myxml}{xml}{tabsize=2,fontsize=\footnotesize}
\mynewminted{myshell}{shell-session}{tabsize=2,fontsize=\footnotesize}
\mynewminted{myjson}{json}{tabsize=2,fontsize=\footnotesize}
\mynewminted{mycode}{text}{tabsize=2,fontsize=\footnotesize}

\newtcblisting[auto counter,number within=section,
  list inside=mypyg]{listingsbox}[3][]{%
  listing only,title={Listing \thetcbcounter: #3},
  list entry={\protect\numberline{\thetcbcounter}#3},
  enhanced,breakable,drop fuzzy shadow,myminted/#2,#1}
  
\renewcommand\listoflistingscaption{Codeverzeichnis}

% cref für minted listings

\makeatletter
\crefname{tcb@cnt@listingsbox}{Listing}{Listings}
\Crefname{tcb@cnt@listingsbox}{Listing}{Listings}
\makeatother

% Definition der Kopf- und Fußzeilen
\pagestyle{scrheadings}
\automark[chapter]{chapter}
\rohead{\titel}
\cohead{}
\lohead{\headmark}
\rofoot{\autor}

\makeglossaries							% Abkürzungsverzeichnis erstellen
\input{Content/abkuerzungen}					% Datei mit Abkürzungen laden
\input{Content/glossar}					% Glossar laden

% Mathe nur in Displaymode anzeigen
\everymath{\displaystyle}

% -------------------------------------------------------------------------------------------
%                     Beginn des Dokumenteninhalts
% -------------------------------------------------------------------------------------------
\begin{document}
\setcounter{secnumdepth}{3}					% Nummerierungstiefe für's Contentsverzeichnis
\setcounter{tocdepth}{2}					% Nummerierungstiefe für's Contentsverzeichnis
\sffamily									% für die Titelei serifenlose Schrift verwenden

% ------------------------------ Titelei -----------------------------------------------------

\include{Content/titelseite} 				% erzeugt die Titelseite
\pagenumbering{Roman}						% große, römische Seitenzahlen für Titelei
\include{Content/erklaerung} 				% Einbinden der eidestattlichen Erklärung

\include{Content/abstract}

\tableofcontents							% Erzeugen des Inhalsverzeichnisses
\deftranslation[to=German]{Acronyms}{Abk\"urzungsverzeichnis}
\glsaddall[types={acronym}]					% Alle Akronyme ausgeben
\printglossary[type=\acronymtype]						% Erzeugen des Abkürzungsverzeichnisses
\listoffigures 								% Erzeugen des Abbildungsverzeichnisses 
\listoftables 								% Erzeugen des Tabellenverzeichnisses
\tcblistof[\addchap*]{mypyg}{Codeverzeichnis} % Erzeugen eines Verzeichnisses für Code
\addcontentsline{toc}{chapter}{Codeverzeichnis}
\todototoc
\listoftodos
\clearpage

% --------------------------------------------------------------------------------------------
%                    Content der Bachelorarbeit
%---------------------------------------------------------------------------------------------
\pagenumbering{arabic}						% arabische Seitenzahlen für den Hauptteil				
\rmfamily
\onehalfspacing
% Schusterjungen und Hurenkinder 
%\clubpenalty = 10000
%\widowpenalty = 10000 \displaywidowpenalty = 10000
\interfootnotelinepenalty=10000 % Keine FUßnoten umbrechen
\include{Content/einleitung}
\clearpage
\include{Content/grundlagen}
\clearpage
\include{Content/konzept}
\clearpage
\chapter{Implementierung}

\section{ToDo}

Um ToDos anzuzeigen können alle Befehle des Packages \enquote{todonotes}\footnote{\url{http://www.ctan.org/tex-archive/macros/latex/contrib/todonotes/}} genutzt werden.
Zur Einfachheit können folgende Befehle genutzt werden:

\begin{verbatim}
\mytodo{Kapitel über Krokodile schreiben!}
\mytodoimprove{Neue Quellen raussuchen!}
\end{verbatim}

\mytodo{Kapitel über Krokodile schreiben!}
\mytodoimprove{Neue Quellen raussuchen!}

\section{Literaturverzeichnis}

Das Literaturverzeichnis generiert die Einträge nach DIN 1502-2. Die Styles sind in der alphadin.bst\footnote{\url{http://mirror.unl.edu/ctan/biblio/bibtex/contrib/german/din1505/alphadin.bst}} zu finden.

Beispiele \cite{web:wiki:latex,book:komascript}

\section{Verzeichnisse}

Dieses Template erstellt einige Verzeichnisse, wie sie entsprechend der DHBW-Richtlinien benötigt werden: Inhaltsverzeichnis, Abbildungsverzeichnis, Tabellenverzeichnis, Abkürzungsverzeichnis und Codeverzeichnis

Zur Demonstration einige Beispiele:

\begin{figure}[H]
\centering
\includegraphics[width=\textwidth]{Images/placeholder.png}
\caption{Ich bin ein tolles Bild!}
\end{figure}

\begin{table}[H]
\centering
\begin{tabular}{ll}
	\textbf{Kopfspalte 1} & \textbf{Kopfspalte 2} \\ \hline\hline
	Ich bin               & eine tolle Tabelle!   \\ \hline
	Bin ich               & nicht wundervoll?
\end{tabular}
\caption{Eine tolle Tabelle}
\end{table}

\section{Source code}

Gerade im IT-Bereich kommt es oft vor, dass Source eingebunden werden soll. Natürlich schön mit Syntax-Highlighting. 
Dieses Template unterstütz dies mit Hilfe von minted\footnote{\url{http://www.ctan.org/tex-archive/macros/latex/contrib/minted/}} und pygments\footnote{\url{http://pygments.org/}}. Die Einrichtung dafür ist beschrieben unter \url{http://www.manuel-rauber.com/post/2013/10/28/My-LaTeX-Environment}.
Des Weiteren ist das Template in der Lage, lange Zeilen umzubrechen\footnote{Allerdings hat die Funktion noch einen Bug: \url{http://tex.stackexchange.com/questions/129383/break-lines-in-minted-code}. Bei bereits eingerücktem Code werden die \enquote{Umbruchpfeile} falsch dargestellt.}, sowie Captions und Boxen anzuzeigen. Zu lange Code-Stücke werden automatisch auf die nächste Seite umgebrochen, was man auch visuell sehen kann. Hier für ein längeres Code-Beispiel:

\begin{listingsbox}{mycsharp}{Beispiel-Code}
using System;
using System.IO;
using System.Net;
using System.Net.Http;
using System.Net.Http.Formatting;
using System.Net.Http.Headers;
using System.Text;
using System.Threading.Tasks;

namespace ShortUrl.Core.Formatters
{
	public class TextMediaTypeFormatter : MediaTypeFormatter
	{
		public TextMediaTypeFormatter()
		{
			SupportedEncodings.Add(Encoding.UTF8);
		}

		public override Task WriteToStreamAsync(Type type, object value, Stream writeStream, HttpContent content,
			TransportContext transportContext)
		{
			return Task.Factory.StartNew(() =>
			{

				var str = (string)value;

				using (var textWriter = new StreamWriter(writeStream))
				{
					textWriter.Write(str);
				}
			});
		}

		public override bool CanReadType(Type type)
		{
			return false;
		}

		public override bool CanWriteType(Type type)
		{
			return typeof(string) == type;
		}
	}
}
\end{listingsbox}

\section{Glossar}

Der Aufbau eines Glossar wird entsprechend durch das Package \enquote{glossaries}\footnote{\url{http://www.ctan.org/tex-archive/macros/latex/contrib/glossaries/}} unterstützt.

Du wolltest schon immer mal wissen, was ein \gls{Instrument} ist? ;-)

\clearpage
\include{Content/ergebnis}

% ------------------------------- Anhang ---------------------------------------------------------

\appendix
\clearpage
\pagenumbering{Roman}						% römische Seitenzahlen für Anhang

% Bibliographie nach DIN 1502-2
\bibliographystyle{alphadin} 
\bibliography{Bibliography}
\clearpage
\printglossary
\end{document}